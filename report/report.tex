\documentclass{article}

% Language setting
% Replace `english' with e.g. `spanish' to change the document language
\usepackage[english]{babel}

% Set page size and margins
% Replace `letterpaper' with `a4paper' for UK/EU standard size
\usepackage[letterpaper,top=2cm,bottom=2cm,left=3cm,right=3cm,marginparwidth=1.75cm]{geometry}

% Useful packages
\usepackage{amsmath}
\usepackage{graphicx}
\usepackage[colorlinks=true, allcolors=blue]{hyperref}

\title{Aprendizado de Máquina \\ Trabalho Prático 1}
\author{Luís Felipe Ramos Ferreira \\ 2019022553}

\begin{document}
\maketitle

\section{Introdução}

O Trabalho Prático 1 da disciplina de Aprendizado de Máquina teve como tema a criação de uma rede neuronal para classificação de dígitos escritos a mão, mais especificamente o conhecido conjunto de dados MNIST. Os objetivos do trabalho envolviam analisar como o modelo da rede neuronal iria variar na convergência do erro empírico conforme são modificadas diferentes variáveis de configuração da rede, como a taxa de aprendizado, a quantidade de neurônios na camada oculta e diferentes algoritmos de cálculo de gradiente.

\section{Implementação}

A linguagem escolhida para o desenvolvimento do trabalho foi Python (versão 3.10), devida a sua grande variedade de bibliotecas úteis para ciência de dados e aprendizado de máquina. A modelagem da rede neuronal foi feita com o uso da API keras disponibilizada na biblioteca TensorFlow, uma vez que se tratava de uma ferramenta extremamente completa para todos os objetivos do trabalho que permitia grande flexibilidade na modelagem da rede.

Para organizar o ambiente de desenvolvimento, que englobava várias bibliotecas diferentes, foi utilizado o gerenciador de pacotes Anaconda, o que tornou muito mais fácil trabalhar com os pacotes de ciência de dados citados.

\section{Experimentos}

Os experimentos foram realizados sobre uma subparte da base de dados MNIST, disponibilizados no enunciado do trabalho. Essa parte possui um total de 5000 instâncias, que foram divididas em conjunto de treino (70\%) e conjunto de teste (30\%).

Conforme especificado, foram testados e comparados os resultados da rede neuronal na classificação dos dígitos para diferentes parâmetros de modelagem. Mais especificamente, todas as permutações das seguintes configurações foram utilizadas:

\begin{itemize}
    \item Taxa de aprendizado
    \item Número de neurônios na camada oculta
    \item Tamanho do lote
\end{itemize}

\begin{figure}
\centering

\caption{\label{fig:frog}This frog was uploaded via the file-tree menu.}
\end{figure}

\subsection{How to add Tables}

Use the table and tabular environments for basic tables --- see Table~\ref{tab:widgets}, for example. For more information, please see this help article on \href{https://www.overleaf.com/learn/latex/tables}{tables}. 

\subsection{How to add Lists}

You can make lists with automatic numbering \dots

\begin{enumerate}
\item Like this,
\item and like this.
\end{enumerate}
\dots or bullet points \dots
\begin{itemize}
\item Like this,
\item and like this.
\end{itemize}

\section{Análise dos resultados}

\section{How to add Citations and a References List}

You can simply upload a \verb|.bib| file containing your BibTeX entries, created with a tool such as JabRef. You can then cite entries from it, like this: \cite{greenwade93}. Just remember to specify a bibliography style, as well as the filename of the \verb|.bib|. You can find a \href{https://www.overleaf.com/help/97-how-to-include-a-bibliography-using-bibtex}{video tutorial here} to learn more about BibTeX.

If you have an \href{https://www.overleaf.com/user/subscription/plans}{upgraded account}, you can also import your Mendeley or Zotero library directly as a \verb|.bib| file, via the upload menu in the file-tree.

\bibliographystyle{alpha}
\bibliography{sample}

\end{document}